\documentclass{article}

\usepackage{hyperref}
\usepackage{multicol}
\usepackage{calc,pict2e,picture}
\usepackage{textgreek,textcomp,gensymb,stix}

\title{Test Plan}
\begin{document}
  \maketitle

  \section*{Description}
The Signaler project calculates prime numbers in increasing order and outputs one number per second.

  \section*{Automated Test Cases}
  Automated  tests  may  be  executed  with  make check.  This  will  both  build  the  tests  and  run  them.  These
are all unit tests of functions.

  \section*{Common Test Cases}
  Prerequisites: Signlaer Directory does not already exist
  \subsection*{TC1: Installation}
  1. git clone git@git.umbc.tc:tdqc/tdqc12/rsoto/signaler.git\\
  2. cd Signaler\\
	Expected: signaler directory is created\\
  \\Prerequisites: In the projects directory
  \subsection*{TC2: Correct branch}
  1. git branch\\
Expected: main is default branch\\
  \subsection*{TC3: Build Cleaning}
  1. make\\
2. make clean\\
3. ls *.o\\
Expected: No such file or directory error\\
  \\Prerequisites: signaler Directory does not already exist
  \section*{"Project" Test Cases}
  Prerequisites: In the projects directory; make clean
  \subsection*{TC4: Build Explicit}
1. make signaler\\
2. ls signaler\\
Expected: file is listed\\

\subsection*{TC5: Build Implicit}
1. make\\
2. ls signaler\\
Expected: file is listed\\
\subsection*{TC6: Build Debugging}
1. make debug\\
2. readelf --debug-dump=decodedline signaler\\
Expected: debugging info is listed\\
\subsection*{TC7: Build Cleaning}
1. make\\
2. make clean\\
3. ls signaler\\
Expected: No such file or directory error\\
\\
Prerequisites: In the project's directory; make\\
\subsection*{TC8: No command line arguments}
1. ./signaler\\
2. echo \$?\\
Expected: program exits with proper output \\
\subsection*{TC9: with command line arguments}
1. ./signaler arguments\\
2. echo \$?\\
Expected: program exits with usage error\\
\subsection*{TC10: with s flag set}
1. ./signaler -s num\\
2. echo \$?\\
Expected: program exits with proper output and started at value num instead of 2\\
\subsection*{TC11: with r flag set}
1. ./signaler -r\\
2. echo \$?\\
Expected: program exits with proper output in reverse\\
\subsection*{TC12: with e flag set}
1. ./signaler -e num\\
2. echo \$?\\
Expected: program exits with proper output and ends when num value is surpassed\\
\subsection*{TC13: with t flag set}
1. ./signaler -t num\\
2. echo \$?\\
Expected: program exits with proper output with each number output in num intervals\\
\\



\end{document}




